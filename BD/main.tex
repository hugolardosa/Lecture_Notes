% load lecture note class
\documentclass{easyclass}
\usepackage[portuguese]{babel}
\usepackage{tcolorbox}
\begin{document}
\begin{titlepage}
    \university{Great University}
    \courseid{CS 000}
    \title{Lecture Note}
    \author{Erkang}
    \version{2021 Março}
    \instructor{}
    \maketitle
\end{titlepage}

\tableofcontents
\clearpage

\section{Introdução aos sistemas de bases de dados}

\subsection{Conceito de Base de dados}
Uma base de dados é uma coleção organizada de dados que estão relacionados e que podem ser partilhados com múltiplas aplicações\\
\begin{tcolorbox}[
colframe=blue!25,
colback=blue!10,
coltitle=blue!20!black,  
fonttitle=\bfseries,
adjusted title=Evolução histórica]

\begin{itemize}
\item 50s - 60s   -   Processamento de dados isolados
\item 60s - 80s   -   Sistema de Gestão de Ficheiros
\item 80s - Hoje  -   Base de dados
\end{itemize}





\end{tcolorbox}

\subsection{Processamento de dados isolados}
No conceito de dados isolados cada aplicação gere os seus próprios dados, estes podem estar replicados, terem diferentes mas isto trás incoerências (problemas de "sincronismo")
\subsection{Sistema de Gestão de Ficheiros}

%\bibliography{bibfile}

\end{document}