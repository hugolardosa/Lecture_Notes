% load lecture note class
\documentclass{easyclass}
\begin{document}
\begin{titlepage}
    \university{Universidade de Aveiro}
    \courseid{AR}
    \title{Apontamentos de AR}
    \author{Hugo Leal}
    \version{2021 Março}
    \instructor{Baseado nos slides da disciplina}
    \maketitle
\end{titlepage}

\tableofcontents
\clearpage

\section{Enterprise Network Design Topics}
É a base do desenho de arquitetura de uma rede 
\subsection{Aspetos fundamentais de uma rede}
\begin{itemize}
    \item A rede deve ser modular (possibilidade de colocar e tirar coisas da rede com o mínimo impacto) [Ex. Numa pandemia é desligado os switches que ligam todos os PCs uma vez que os trabalhadores estão em teletrabalho]
    \begin{itemize}
        \item Suportar crescimento/decréscimo e mudança
        \item Mudar o tamanho de uma rede é mais fácil adicionando novos módulos do que fazer um completo redisign
    \end{itemize}
    \item A rede deve ser resiliente
    \begin{itemize}
        \item Estár sempre "up" ("up-time" próximo de 100\%)
        \begin{itemize}
            \item A falha de uma rede financial pode representar milhões em perda
            \item A falha de rede num hospital pode fazer peder vidas
        \end{itemize}
        \item Mas a resiliência tem custos
        \begin{itemize}
            \item O nível de resiliência deve ser um "trade-off" entre o orçamento disponível e um risco aceitavel  
        \end{itemize}
    \end{itemize}
    \item A rede deve ser flexível
    \begin{itemize}
        \item Mercados mudam e desenvolvem-se $ \mapsto $ A rede deve se adaptar rapidamente
        \item Num contexto pandémico uma empresa de serviços a rede tem que se adaptar ao teletrabalho
    \end{itemize}
\end{itemize}
\subsection{Equipamentos}
\begin{itemize}
    \item [Switches] Switches L2, Wireless Access Points (Diff. Velocidades, Protocolos)
    \item [Routers] Layer3 com funcionalidades como QoS, Segurança, VPN gatways, Monitorização de redes (Hoje em dia grande parte destas funcionalidades são separadas para diferentes aparelhos)
    \item [L3 Switch] Switch que já faz routing basico (Sw em hardware com Routing em software)
    \item [Router com Sswitching] (Sw em Software com Routing em hardware)
    \item [Security Appliances (Modulares)] 
    \begin{itemize}
        \item Firewalls (Regras e controlo do que passa e o que não passa)
        \item IDS/IPS (Verificam anomalia no tráfego)
        \item NAT/PAT(costuma ser feito na Firewall, para facilitar aplicação de regras)
        \item VPN Gateway (Implementados no router ou na Firewall)
        \item Services Proxy (Vídeos/Páginas mais comum ficam guardadas em cache.  Mais comum em empresas multimédia)
    \end{itemize}
    
\end{itemize}
\subsection{Como escolher equipamentos}
\begin{itemize}
    \item Tipo
    
    \begin{itemize}
        \item L2 Switch, L3 Switch, Router, \dots, etc.
    \end{itemize}

    \item \textit{Manufactor}
     
    \begin{itemize}
        \item Viabilidade
        \begin{itemize}
            \item Tentar obter o máximo de \textit{MTBF (Mean Time Betweebn Faliures)}
        \end{itemize}
    \end{itemize}

    \item Range/Model
    
    \begin{itemize}
        \item Quantas portas são precisas
        \item Numero de portas
        \begin{quotation}
           [Projeto] Costumam ter 2 portas uma para resiliência outra para controlo remoto (As duas ligadas)
        \end{quotation}

        \begin{itemize}
            \item Protocolos e funcionalidades suportadas
            \item Determinação de requisitos de memória
        \end{itemize}
        \item Velocidade de processamento e de comunicação
        \begin{itemize}
            \item Numero de bytes/pacotes processados/comutados por segundo
        \end{itemize}
    \end{itemize}

\end{itemize}

%\bibliography{bibfile}

\end{document}